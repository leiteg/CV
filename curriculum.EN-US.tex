%----------------------------------------------------------------------------------------
%	PACKAGES AND OTHER DOCUMENT CONFIGURATIONS
%----------------------------------------------------------------------------------------

\documentclass{resume}

\usepackage[left=0.75in,top=0.6in,right=0.75in,bottom=0.6in]{geometry}
\usepackage[T1]{fontenc}
\usepackage[english]{babel}
\usepackage{microtype}
\usepackage{lmodern}
\usepackage{enumitem}
\usepackage{multicol}
\usepackage{booktabs}

\newcommand{\tab}[1]{\hspace{.2667\textwidth}\rlap{#1}}
\newcommand{\itab}[1]{\hspace{0em}\rlap{#1}}

\name{Gustavo Leite}
\address{Av. Albert Einstein, 1251, Cidade Universitária, Campinas-SP, Brazil, ZIP 13083-852}
\address{+55 (19) 99721-4443 \\ {\tt contact@gustavoleite.me}}

\begin{document}

%----------------------------------------------------------------------------------------
%	EDUCATION SECTION
%----------------------------------------------------------------------------------------

\begin{rSection}{Education}
{\bf BSc in Computer Science} \hfill {\em 2013--2016} \\
{\sc São Paulo State University (UNESP)} \hfill {\em Rio Claro-SP, Brazil} \\
I finished the course with grade 8.58 out of 10, placed as the first student of
the class. I developed a study on parallel computing with application to the
$n$-body problem as my final paper. This work was supervised by Prof. Alexandro
Baldassin.

{\bf MSc in Computer Science} \hfill {\em 2017--2019} \\
{\sc São Paulo State University (UNESP)} \hfill {\em Rio Claro-SP, Brazil} \\
I obtained grade A in 5 out of 6 courses attended. I defended the thesis titled
``Performance Evaluation of Code Optimizations in FPGA Accelerators'' in August,
2019 under supervision of Prof. Alexandro Baldassin.

{\bf Complementary Training} \hfill {\em 2018} \\
{\sc University of Alberta} \hfill {\em Edmonton-AB, Canada} \\
During a research internship, I audited two courses on compilers in the
University of Alberta. The courses were ``Compiler Design'' and ``Machine
Learning in Optimizing Compilers'', both lectured by Prof. José Nelson Amaral
during the Fall of 2018.

{\bf PhD in Computer Science} \hfill {\em 2020--present} \\
{\sc University of Campinas (Unicamp)} \hfill {\em Campinas-SP, Brazil} \\
I am currently a PhD student at the University of Campinas, working on the
Computer Systems Laboratory (LSC) under the supervision of Prof. Guido Araújo
and Prof. Marcio Machado Pereira.
\end{rSection}

%----------------------------------------------------------------------------------------
%	SKILLS SECTION
%----------------------------------------------------------------------------------------

\begin{rSection}{Awards and Distinctions}
{\bf Best Academic Performance} $\diamond$ {UNESP} \hfill{\em 2017} \\
In March 2017, I was awarded a distinction from the São Paulo State University
for the best academic performance in the Bachelors in Computer Science 2016's
class.

{\bf Winner of the Parallel Programming Challenge} $\diamond$ {ERAD-SP, ICMC-USP} \hfill{\em 2017} \\
In April 2017, my team won the Parallel Programming Challenge carried during the
8th Regional School of High-Performance (ERAD-SP), ICMC-USP, São Carlos-SP,
Brazil.
\end{rSection}

\begin{rSection}{Interests}
Following, a non-exhaustive list of my professional and research interests:

\setlength\multicolsep{5pt}
\begin{multicols}{2}
\begin{itemize}[noitemsep]
  \item high-performance computing;
  \item compilers;
  \item cloud computing;
  \item computer architecture.
\end{itemize}
\end{multicols}

\end{rSection}

%----------------------------------------------------------------------------------------
%	EXPERIENCE SECTION
%----------------------------------------------------------------------------------------

\begin{rSection}{Professional Experience}
{\it I do not have professional experience.}
\end{rSection}

%--------------------------------------------------------------------------------
%    PROJECTS
%-----------------------------------------------------------------------------------------------

\begin{rSection}{Research Experience}
{\bf Self-Healing Software} \hfill {\em 2014--2015} \\
{\sc São Paulo State University (UNESP)} \hfill {\em Rio Claro-SP, Brazil} \\
During the bachelors course, I have participated in two research projects aimed
at developing better recommendation systems to assist decision-making in
self-healing systems. I was supervised by Prof. Frank José Affonso and received
funding from CNPq and FAPESP. During this period I coauthored two papers
published in international conferences~\cite{SEKE16,SEKE15}.

{\bf Parallel Programming} \hfill {\em 2016} \\
{\sc São Paulo State University (UNESP)} \hfill {\em Rio Claro-SP, Brazil} \\
I have developed a term paper where I studied the parallel programming paradigm
using OpenMP and OpenCL. With the knowledge obtained, I compared the performance
of implementations in CPU and GPU of the $n$-body problem. This work
was supervised by Prof. Alexandro Baldassin.

{\bf High-Performance Computing} \hfill {\em 2017--2019} \\
{\sc São Paulo State University (UNESP)} \hfill {\em Rio Claro-SP, Brazil} \\
During the masters course, I have investigated workload balancing techniques for
NUMA systems and, after that, I have analyzed the performance of code
optimizations aimed at FPGA accelerators present in the literature. I defended
the thesis titled ``Performance Evaluation of Compiler Optimizations in FPGA
Accelerators''~\cite{Master19} in August 2019 under the supervision of Prof.
Alexandro Baldassin. This work was carried with funding from CAPES and FAPESP.
During this period I have also been a teacher assistant in the undergrad course
``Microprocessors II''.


{\bf High-Performance Computing} \hfill {\em 2018} \\
{\sc University of Alberta} \hfill {\em Edmonton-AB, Canada} \\
I visited the University of Alberta between September 2018 and November 2018 for
a research internship.  The objetive of this project was to conduct a
bibliographic survey on existing compiler optimizations for FPGA accelerators. I
was co-supervised by Profs. José Nelson Amaral (UAlberta) and Guido Araújo
(IC-Unicamp). This internship was funded by the BEPE/FAPESP program. From this
collaboration, we published a paper in a brazilian conference~\cite{WSCAD19}.
\end{rSection}


%----------------------------------------------------------------------------------------
%	SKILLS SECTION
%----------------------------------------------------------------------------------------

\begin{rSection}{Skills and Language}
{\bf Programming Languages}

\begin{tabular}{ll}
  Over 5K lines & C, C++, Java, Python \\
  Over 1K lines & OpenCL, Gnuplot, ANTLR4 grammar \\
  Familiar      & Assembly, AWK, PHP, Bash, CUDA, Javascript \\
\end{tabular}

{\bf Tools} {\small (Ordered by familiarity)}
\begin{itemize}[noitemsep]
  \item Linux operating system and utilities ({\tt grep}, {\tt bash}, {\tt awk},
    {\tt make}, etc);
  \item C/C++ compilers ({\tt gcc} and {\tt clang});
  \item Control version system ({\tt git}) and platforms (Github, Gitlab, etc);
  \item Parallel programming paradigms (OpenMP, OpenCL);
  \item Python packages ({\tt numpy}, {\tt matplotlib}, {\tt seaborn});
  \item Debuggers ({\tt gdb}) and profilers ({\tt perf}, {\tt strace}, {\tt
    ltrace});
  \item LLVM compiler infrastructure.
\end{itemize}

{\bf Languages Proficiency}
\begin{itemize}[noitemsep]
  \item {\bf Portuguese} (Native) -- comprehension: good; speaking: good; writing: good;
  \item {\bf English} (Fluent) -- comprehension: good; speaking: good; writing: good;
  \item {\bf German} (Beginner/A2) -- comprehension: basic; speaking: none; writing: basic;
\end{itemize}
\end{rSection}

\begin{rSection}{Event Attendance}
{\bf SBAC-PAD'17} $\diamond$ Campinas-SP, Brazil \hfill {\em 2017} \\
International Symposium on Computer Architecture and High Performance
Computing

{\bf 8ª ERAD-SP} $\diamond$ ICMC-USP, São Carlos-SP, Brazil \hfill {\em 2017} \\
Escola Regional de Alto Desempenho do Estado de São Paulo

{\bf SBAC-PAD'19} $\diamond$ Campo Grande-MS, Brazil \hfill {\em 2019} \\
International Symposium on Computer Architecture and High Performance
Computing
\end{rSection}

\begin{rSection}{Links}
  \begin{tabular}{ll}
    Github            & {\tt https://github.com/leiteg} \\
    Gitlab            & {\tt https://gitlab.com/leiteg} \\
    Twitter           & {\tt https://twitter.com/gstvleite} \\
    Google Scholar    & {\tt https://scholar.google.com/citations?user=F6MZj\_oAAAAJ} \\
    Lattes Curriculum & {\tt https://lattes.cnpq.br/0392351138118593} \\
    LSC Homepage      & {\tt https://lsc.ic.unicamp.br/}
  \end{tabular}
\end{rSection}

\begin{rSection}{Publications}
\bibliographystyle{abbrv}
\renewcommand{\section}[2]{}
\bibliography{references}
\end{rSection}

\vfill \hfill {\em Last updated: \today.}

\end{document}
