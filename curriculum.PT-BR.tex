%----------------------------------------------------------------------------------------
%	PACKAGES AND OTHER DOCUMENT CONFIGURATIONS
%----------------------------------------------------------------------------------------

\documentclass[a4paper]{resume}

\usepackage[left=0.75in,top=0.6in,right=0.75in,bottom=0.6in]{geometry}
\usepackage[T1]{fontenc}
\usepackage[brazil]{babel}
\usepackage{microtype}
\usepackage{lmodern}
\usepackage{enumitem}
\usepackage{multicol}
\usepackage{booktabs}

\newcommand{\tab}[1]{\hspace{.2667\textwidth}\rlap{#1}}
\newcommand{\itab}[1]{\hspace{0em}\rlap{#1}}

\name{Gustavo Leite}
\address{Av. Albert Einstein, 1251, Cidade Universitária, Campinas-SP, Brasil, CEP 13083-852}
\address{+55 (19) 99721-4443 \\ {\tt contact@gustavoleite.me}}

\begin{document}

%----------------------------------------------------------------------------------------
%	EDUCATION SECTION
%----------------------------------------------------------------------------------------

\begin{rSection}{Educação}
{\bf Bacharelado em Ciência da Computação} \hfill {\em 2013--2016} \\
{\sc Universidade Estadual Paulista (UNESP)} \hfill {\em Rio Claro-SP, Brasil} \\
Terminei o curso com coeficiente de rendimento 8,58, colocado como o primeiro da
turma. Desenvolvi um trabalho de graduação intitulado ``Estudo sobre Computação
Paralela com Aplicação ao Problema dos $n$-corpos'' sob orientação do Prof.
Alexandro Baldassin.

{\bf Mestrado em Ciência da Computação} \hfill {\em 2017--2019} \\
{\sc Universidade Estadual Paulista (UNESP)} \hfill {\em Rio Claro-SP, Brasil} \\
Terminei o curso com conceito A em 5 das 6 disciplinas cursadas. Defendi a
dissertação intitulada ``{\it Performance Evaluation of Code Optimizations in
FPGA Accelerators}'' em agosto de 2019 sob orientação do Prof. Alexandro
Baldassin.

{\bf Formação Complementar} \hfill {\em 2018} \\
{\sc University of Alberta} \hfill {\em Edmonton-AB, Canadá} \\
Durante um estágio de pesquisa, cursei duas disciplinas sobre compiladores como
ouvinte na University of Alberta. Os cursos ``{\em Compiler Design}'' e ``{\em
Machine Learning in Optimizing Compilers}'' foram ambos ministrados pelo Prof.
José Nelson Amaral entre setembro e dezembro de 2018.

{\bf Doutorado em Ciência da Computação} \hfill {\em 2020--presente} \\
{\sc Universidade Estadual de Campinas (Unicamp)} \hfill {\em Campinas-SP, Brasil} \\
Atualmente estou cursando Doutorado em Ciência da Computação na Universidade
Estadual de Campinas. Trabalho no Laboratório de Sistemas de Computação (LSC)
sob orientação do Prof. Guido Araújo e Prof. Marcio Machado Pereira.
\end{rSection}

%----------------------------------------------------------------------------------------
%	SKILLS SECTION
%----------------------------------------------------------------------------------------

\begin{rSection}{Prêmios e Distinções}
{\bf Melhor Desempenho Acadêmico} $\diamond$ {UNESP} \hfill{\em 2017} \\
Em março de 2017, recebi uma distinção da Universidade Estadual Paulista pelo
melhor desempenho acadêmico no curso de Bacharelado em Ciências da Computação --
Integral -- Turma 2016.

{\bf Vencedor do Desafio de Programação Paralela} $\diamond$ {ERAD-SP, ICMC-USP} \hfill{\em 2017} \\
Em abril de 2017, fui integrante da equipe vencedora do Desafio de Programação
Paralela que ocorreu durante a 8ª Escola Regional de Alto Desempenho (ERAD-SP),
ICMC-USP.
\end{rSection}

\begin{rSection}{Interesses}
A seguir uma lista não exaustiva de interesses profissionais e de pesquisa:

\setlength\multicolsep{5pt}
\begin{multicols}{2}
\begin{itemize}[noitemsep]
  \item computação de alto desempenho;
  \item compiladores;
  \item computação em nuvem;
  \item arquitetura de computadores.
\end{itemize}
\end{multicols}

\end{rSection}

%----------------------------------------------------------------------------------------
%	EXPERIENCE SECTION
%----------------------------------------------------------------------------------------

\begin{rSection}{Experiência Profissional}
{\it Não possuo experiência profissional.}
\end{rSection}

%--------------------------------------------------------------------------------
%    PROJECTS
%-----------------------------------------------------------------------------------------------

\begin{rSection}{Experiência em Pesquisa}
{\bf Software Auto-Curável} \hfill {\em 2014--2015} \\
{\sc Universidade Estadual Paulista (UNESP)} \hfill {\em Rio Claro-SP, Brasil} \\
Durante o curso de graduação participei de dois projetos de pesquisa cujos
objetivos foram desenvolver sistemas de recomendação para auxiliar na tomada de
decisão em sistemas de auto-cura. Recebi financiamento das agências CNPq e
FAPESP e trabalhei sob orientação do Prof. Frank José Affonso. Neste período fui
co-autor de dois artigos publicados em conferências
internacionais~\cite{SEKE16,SEKE15}.

{\bf Programação Paralela} \hfill {\em 2016} \\
{\sc Universidade Estadual Paulista (UNESP)} \hfill {\em Rio Claro-SP, Brasil} \\
Desenvolvi um trabalho de graduação onde estudei o paradigma de programação
paralela usando OpenMP e OpenCL. Com o conhecimento adquirido, realizei uma
avaliação de desempenho comparando implementações do problema dos $n$-corpos
tanto em CPU quando em GPU. Este trabalho foi orientado pelo Prof. Alexandro
Baldassin.

{\bf Computação de Alto Desempenho} \hfill {\em 2017--2019} \\
{\sc Universidade Estadual Paulista (UNESP)} \hfill {\em Rio Claro-SP, Brasil} \\
Durante uma parte curso de mestrado trabalhei pesquisando sobre técnicas de
balanceamento de carga em sistemas NUMA e, na sequência, passei a pesquisar e
avaliar o desempenho de  técnicas de otimização de código para FPGAs presentes
na literatura. Defendi a dissertação intitulada ``{\it Performance Evaluation of
Compiler Optimizations in FPGA Accelerators}''~\cite{Master19} em agosto de
2019. Recebi financiamento das agências CAPES e FAPESP durante o mestrado. Neste
período também fui monitor da disciplina ``Microprocessadores II'' ministrada
para o curso de graduação.

{\bf Computação de Alto Desempenho} \hfill {\em 2018} \\
{\sc University of Alberta} \hfill {\em Edmonton-AB, Canadá} \\
Visitei a University of Alberta no período de setembro de 2018 a novembro de
2018. O projeto de pesquisa envolveu o levantamento bibliográfico de técnicas de
otimização de código direcionadas para aceleradores FPGA. Fui co-orientado pelos
Profs. José Nelson Amaral (UAlberta) e Guido Araújo (IC-Unicamp). Este estágio
de pesquisa foi financiado através do programa BEPE da FAPESP. A partir desta
colaboração um artigo foi publicado em um evento nacional~\cite{WSCAD19}.
\end{rSection}


%----------------------------------------------------------------------------------------
%	SKILLS SECTION
%----------------------------------------------------------------------------------------

\begin{rSection}{Habilidades e Proficiência em Línguas}
{\bf Linguagens de Programação}

\begin{tabular}{ll}
  Escrevi acima de 5000 linhas & C, C++, Java, Python \\
  Escrevi entre 1000 e 5000 linhas & OpenCL, Gnuplot, ANTLR4 grammar \\
  Possuo familiaridade & Assembly, AWK, PHP, Bash, CUDA, Javascript \\
\end{tabular}

{\bf Ferramentas} {\small (Ordenadas por familiaridade)}
\begin{itemize}[noitemsep]
  \item Sistema operacional Linux e programas utilitários ({\tt grep}, {\tt
    bash}, {\tt awk}, {\tt make}, etc);
  \item Compiladores {\tt gcc} e {\tt clang} de C/C++;
  \item Sistema de controle de versão ({\tt git}) e plataformas (Github, Gitlab,
    etc);
  \item Paradigmas de programação paralela OpenMP e OpenCL;
  \item Pacotes {\tt numpy}, {\tt matplotlib}, {\tt seaborn} de Python;
  \item Debuggers ({\tt gdb}) e perfiladores ({\tt perf}, {\tt strace}, {\tt
    ltrace});
  \item Infraestrutura de compiladores LLVM.
\end{itemize}

{\bf Proficiência em Línguas}
\begin{itemize}[noitemsep]
  \item {\bf Português} (Nativo) -- compreendo bem; falo bem; escrevo bem.
  \item {\bf Inglês} (Fluente) -- compreendo bem; falo bem; escrevo bem.
  \item {\bf Alemão} (Iniciante/A2) -- compreendo pouco; escrevo pouco.
\end{itemize}
\end{rSection}

\begin{rSection}{Participação em Eventos}
{\bf SBAC-PAD'17} $\diamond$ Campinas-SP, Brazil \hfill {\em 2017} \\
International Symposium on Computer Architecture and High Performance
Computing

{\bf 8ª ERAD-SP} $\diamond$ ICMC-USP, São Carlos-SP, Brazil \hfill {\em 2017} \\
Escola Regional de Alto Desempenho do Estado de São Paulo

{\bf SBAC-PAD'19} $\diamond$ Campo Grande-MS, Brazil \hfill {\em 2019} \\
International Symposium on Computer Architecture and High Performance
Computing
\end{rSection}

\newpage

\begin{rSection}{Links}
  \begin{tabular}{ll}
    Github         & {\tt https://github.com/leiteg} \\
    Gitlab         & {\tt https://gitlab.com/leiteg} \\
    Twitter        & {\tt https://twitter.com/gstvleite} \\
    Google Scholar & {\tt https://scholar.google.com/citations?user=F6MZj\_oAAAAJ} \\
    Lattes         & {\tt https://lattes.cnpq.br/0392351138118593} \\
    Página do LSC  & {\tt https://lsc.ic.unicamp.br/}
  \end{tabular}
\end{rSection}

\begin{rSection}{Publicações}
\bibliographystyle{abbrv}
\renewcommand{\section}[2]{}
\bibliography{references}
\end{rSection}

\vfill \hfill {\em Atualizado em: \today.}

\end{document}
